The iterative step of our compiler must make a decision about which \verb|par|
setting to provide the runtime with for the next execution of the program. In
the previous chapter we used heuristic techniques to provide this decision
making. Now we will use the notion of \verb|par| site health described in the
previous section.

After every execution of the program, turn off the \verb'par' site with the
lowest health (the lowest average reduction count). In the case of the
execution statistics displayed in Figure \ref{fig:sumHist} we would disable
\verb-par- site \#8\footnote{Notice that \texttt{par} site \#8 also has some
very productive threads, but even so, its mean is the lowest in that run of the
program.}, allowing us to avoid the overhead of all the unproductive threads it
sparked off.  Then repeat this process until switching a \verb'par' site off
increases the overall runtime of the program.


\begin{algorithm}
\DontPrintSemicolon
\TitleOfAlgo{Profile-Driven Search}
\SetKw{Break}{break}
\SetKwData{Setting}{setting}
\SetKwData{Stats}{stats}
\SetKwData{Run}{runtime}
\SetKwData{Pars}{parStats}
\SetKwData{Health}{health}
\SetKwData{Last}{last}
\SetKwData{Cur}{current}
\SetKwArray{Set}{setting}
\SetKwData{Weakest}{weakest}
\SetKwFunction{evaluateProg}{evaluateProg}
\SetKwFunction{CalcH}{calculateHealth}
\SetKwFunction{flipSwitch}{flipSwitch}
\SetKwFunction{Done}{allOff?}

\KwData{An initial \texttt{par} setting as a bitstring of size $N$ (all bits on)}
\KwResult{The best performing bitstring}
\BlankLine

\Last.\Run $\leftarrow \infty$\;
\BlankLine

\For{$i \leftarrow 0$ \KwTo $N$}{
    \Cur $\leftarrow$ \evaluateProg{}\;
    \BlankLine
    \uIf{\Cur.\Run $>$ \Last.\Run}{
        \Break\;
    }
    \ElseIf{\Done{\Cur.\Setting}}{
        \Return \Cur.\Setting\;
    }
    \BlankLine
    \Last $\leftarrow$ \Cur\;
    \BlankLine
    \Weakest $\leftarrow$ \CalcH{}\;
    \BlankLine
    \flipSwitch{\Cur.\Set{\Weakest}}\;
}
\BlankLine
\Return \Last.\Setting\;
\caption{Greedy \texttt{par}-Setting Search}
\label{list:greedy}
\end{algorithm}

It is worth noting that our algorithm is really a hill-climbing algorithm with
an oracle. Instead of randomly evaluating neighbours, the search moves to the
neighbour where the weakest \verb|par| site is switched off. The success of this
search algorithm will depend on how well this corresponds to an increase in
overall performance.
