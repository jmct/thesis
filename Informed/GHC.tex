While the results above are encouraging we would like to see how the resulting
programs perform when compiled by a modern high-performance Haskell compiler.
To do this we extract the final \verb-par- settings from each program and
translate that to Haskell suitable for compilation by GHC.

For the versions parallelised by hand we use the \verb-par- placements found
in the literature \citep{vGMachine, runciman1994profiling}.

\begin{table}[ht]
\caption[Naive transfer to GHC]{Speedups compared to the sequential program as compiled by GHC for
         manually and automatically parallelised versions}
\centering
\smallskip
  \begin{tabular}{ l||c c }
    Program & \multicolumn{2}{c}{4-core} \\
            & Hand   & Auto         \\
    \hline
    SumEuler  & 3.32    & 3.31      \\
    Queens    & 1.76    & 0.97      \\
    Queens2   & 2.29    & 0.61     \\
    SodaCount & 1.25    & 0.64      \\
    Tak       & 1.77    & 1.64      \\
    MatMul    & 1.75    & 0.80      \\
  \end{tabular}
\label{tableGHC}
\end{table}

As Table \ref{tableGHC} makes clear, the results are not impressive. In fact,
except for \verb-SumEuler- and \verb-Tak-, all of the parallel benchmarks
performed \emph{worse} than their sequential counterparts.

However, we feel that not all hope is lost. There are a few recurring issues in
the generated program. A common issue is that the generated strategies will not
be what forces the evaluation of a value. Take the following example as an
illustration

\begin{verbatim}
foo n = let ys = gen n n
        in par (tailStrict1 ys) (bar ys)

tailStrict1 xs = case xs of
    y:ys -> tailStrict1 ys
    []   -> ()
\end{verbatim}

In the function \verb-foo- we spark off a strategy that is meant to force the
spine of the list \verb-ys-, the catch is that GHC's \verb-par- is fast enough
for \verb-bar ys- to be what forces the evaluation of \verb-ys-. So we're
paying the overhead and reaping none of the benefits. In some programs changing
a \verb-par- like the one found in \verb-foo- to a \verb-seq- is enough to solve the
issue and make the parallel version competitive with the manually parallelised version.
