The results from the previous chapter, while promising, assume that the runtime
system is a `black box'. In some cases it may be necessary or desirable to make
this assumption but we feel that with access to information from the runtime
system our technique can produce meaningful results with fewer iterations.

The work in this chapter uses profile information and other statistics from
each iteration of the program before determining where to proceed in the search
space.

\subsection{Plan of the Chapter}

\S\ref{sec:overview} presents a high-level overview of our approach to implicit
parallelism.  \S\ref{sec:defunct} explains the advantages of performing
defunctionalisation on the source program. \S\ref{sec:strictness} motivates our
use of a projection-based strictness analysis.  \S\ref{sec:proAndStrat}
describes the correspondence between projections and strategies which allows us
to generate parallel strategies based on the projections provided by the
strictness analysis. \S\ref{sec:introPar} describes the initial placement of
\verb-par- annotations in the source program. \S\ref{sec:iterate} introduces
the technique used for utilising the runtime profiling to disable some of the
introduced parallelism along with possible additional search techniques.
\S\ref{sec:results} presents some experimental results and discussion of those
results. Lastly, \S\ref{sec:conclusion} contains our conclusions and thoughts
on possible future work.
