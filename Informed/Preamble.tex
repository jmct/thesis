The results from the previous chapter, while promising, assume that the runtime
system is a `black box'. In some cases it may be necessary or desirable to make
this assumption but we feel that with access to information from the runtime
system our technique can produce meaningful results with fewer iterations.

The work in this chapter uses profile information and other statistics from
each iteration of the program before determining where to proceed in the search
space.

\subsection*{Plan of the Chapter}

We begin by defining the concept of \verb|par| site health in Section
\ref{sec:parHealth}.  This metric provides us with a way to make decisions
about how to proceed at each iterative step. The algorithm for incorporating
\verb|par| site health is presented in Section \ref{sec:search1}. We then
present and discuss the results of using this algorithm in Section
\ref{sec:infResults}, we use the opportunity of having profiling information to
also experiment with various simulated overheads for the `cost' of a
\verb|par|. In Section \ref{sec:ghcComp} show how the benchmark programs
perform when we naively transfer the resulting programs to be compiled by GHC.
Lastly, we provide a summary of our results from these experiments and some
conclusions that we can draw from them in Section \ref{sec:informedConclusion}.
