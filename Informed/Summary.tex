This chapter has provided a method to utilise runtime profile data in order to
better search for an optimal \verb|par| setting of a given program. The use
of profile information gives the hill-climbing algorithm an oracle that
predicts which neighbor will be the most performant. The oracle is based
on the concept of \verb|par| site health, the mean of the work undertaken
by all threads sparked by a \verb|par|. This proves to be a good metric
for our runtime system but does not translate naively to GHC.

The naive transfer to GHC is a disappointing result. That being said, iterative
compilation techniques are usually performed on the same substrate as the final
program. Using one runtime system (a simulator) and then transferring the
results to another, completely different, runtime system proves to be too
optimistic and does not correspond to the performance increases we would like.

We feel that performing the iterative step on GHC itself would likely
provide better results, but testing this hypothesis requires adapting the
compiler, static analysis, and runtime system of GHC in non-trivial ways.

In addition to performing the iteration on GHC itself it may be beneficial to
abandon the bitstring representation of a program's \verb|par| settings. In
Section \ref{sec:ghcComp} we note that many of the performance issues when
transferring to GHC can be fixed by switching certain \verb|par|s to
\verb|seq|s. We may be able to adapt the search to begin by turning off the
\emph{obviously bad} \verb|par|s and then perform the remaining search with
each \verb|par| site having three modes: on, off, and \verb|seq|.

\subsection{Comparison to Heuristic Search}

In the previous chapter we explored using two heuristic search algorithms to
accomplish our iterative step. While the results were promising, the ability to
examine profile data should not be underestimated. The first benefit is that
our search is guaranteed to be linear \emph{in the worst case} and usually
sub-linear, whereas even the greedy algorithm required exactly linear time.

As the programs grow larger the combinatorial explosion for the hill-climbing
will become more and more detrimental. For the hill-climbing algorithm to
terminate all neighbors to the current candidate must be explored. This means
that when a program has twenty \verb|par| sites, the \emph{last} iteration of
the hill-climbing algorithm will require twenty evaluations of the program! The
profile based technique's ability to guarantee that this program will require
\emph{at most} twenty iterations is a huge advantage. This drastically reduces
the cost of finding the worthwhile parallelism.
