The thesis is divided into three broad parts:

Part \ref{part:idea} explores the central concepts and the overall idea.  In
Chapter \ref{chap:background} we review the work on parallel functional
programming, discussing the benefits and drawbacks to different approaches of
writing (or not writing!) parallel programs.  Chapter \ref{chap:overview}
provides an overview of our technique and a standard vocabulary for the rest of
the work.  

Part \ref{part:static} is devoted to the \emph{static} aspects of our approach.
This includes a review of strictness analysis and the motivation with utilising
a projection-based analysis in Chapter \ref{chap:discovery}. We present our
technique for exploiting the results of strictness analysis in Chapter
\ref{chap:derivation}.

In Part \ref{part:implementation} we first describe our experimental platform
in Chapter \ref{chap:platform}, then discuss two experiments: Chapter
\ref{chap:blind} experiments with using heuristic search techniques based on
the overall runtime of a program and in Chapter \ref{chap:prof-search} we
provide the compiler with access to more detailed runtime profiles. 

Lastly, Part \ref{part:conclusion} discusses our conclusions (Chapter
\ref{chap:conclusions}) and possible future work (Chapter \ref{chap:future}).
