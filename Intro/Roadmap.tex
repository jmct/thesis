The thesis is divided into three broad parts:

We begin with Part \ref{part:idea} where we explore the overall idea in Chapter
\ref{chap:overview} to provide a standard vocabulary for the rest of the work.
In Chapter \ref{chap:background} we review the work on parallel functional
programming, discussing the benefits and downsides to different approaches of
writing (or not writing!) parallel programs.

Part \ref{part:static} is devoted to the \emph{static} aspects of our approach.
This includes a review of strictness analysis and the motivation with utilising
a projection-based analysis in Chapter \ref{chap:discovery}. We present our
technique for exploiting the results of strictness analysis in Chapter
\ref{chap:derivation}.

In Part \ref{part:implementation} we first describe our experimental platform
in Chapter \ref{chap:platform}, then discuss two experiments, Chapter
\ref{chap:blind} experiments with using hueristic search techniques based on
the overall runtime of a program and in Chapter \ref{chap:prof-search} we
provide the compiler access to more detailed runtime profiles. 

Lastly, in Part \ref{part:conclusion} we discuss possible future work (Chapter
\ref{chap:future}) and the conclusions we have arrived to (Chapter
\ref{chap:conclusions}).
