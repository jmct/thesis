Using the projection-based strictness analysis to discover producer-consumer
pairs we may be able to automatically utilise pipeline parallelism. Because of
lazy evaluation we already benefit form a form of pipeline parallelism
\citep{whyFPmatters}. However, because parallel expressions are not tied to
specific execution contexts \todo{we're gong to have to define this in the
background chapters} the producer and consumer threads can easily interrupt
each other due to the runtime system scheduler. To prevent this we can use
ideas from Hudak's \emph{para-functional programming} that allows for
expressions to be tagged with operational information, such as where in memory
the expression should allocate data or which processors an expression should be
tied to \citep{hudak1986functional}. If the functions are strict enough we could
employ the techniques introduced by Marlow et al. in the Par monad, which
automatically schedules pipeline parallelism for values that can be fully
evaluated \citep{marlow2011monad}.
