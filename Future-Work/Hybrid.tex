We have presented our work as being fully automated and requiring no
intervention by the programmer, but if we loosen that requirement on
our technique we may be able to find a fruitful `middle-ground' between
fully-automated parallelism and programmer effort.

\subsubsection{Super Strategies}

One technique that we find promising is the idea of an automatic strategy.

It is common for a programmer to know that an algorithm should be
parallelisable but does not want to invest the effort in parallelising the code
by hand. A `super-strategy' would allow the programmer to annotate the program,
telling the compiler `this expression should be parallelisable', but without
specifying how. This saves the compiler from searching for parallelism
throughout the whole program and iterating over the large resulting search
space and instead focus its static analysis and iteration to the annotated
expression.

This could be exposed to the programmer with an interface similar to parallel
strategies \citep{strategies}.

Something along the lines of:

\begin{haskell}
\hsinf{expression Alice would like to parallelise} `using` autoStrat
\end{haskell}

Or more conventionally:

\begin{haskell}
autoPar \hsinf{expression Alice would like to parallelise}
\end{haskell}

The auto-strategy would still use the demand information available at compile
time, and proceed in the manner outlined in this thesis, but the compiler would
benefit from a much reduced search space.
