In our implementation we specialised functions in two ways: higher-order to
first-order and \verb|par| placement based on varied demands. There are of
course, many forms of specialisation that could be added, monomorphisation,
call-pattern specialisation, and even specialisation to some depth of a
recursive function. 

%When dealing with parallel programs, it is easy to think of
%examples where a polymorphic function may be suitable for parallelism when
%instantiated on one type, but not another. This is particularly true of bounded
%polymorphism in the form of typeclasses as found in Haskell
%\citep{haskellReport}. An overloaded \<size\> function may benefit from having
%its argument evaluated in parallel for one type (binary trees for example) and
%not for another
%
% I don't think the above idea holds water... If we have an overloaded function
% than it's individual instances are the things that will have the parallelism
% and even if not. The calls to `size` will _each_ have their own par sites,
% which can be turned off individually.

When writing parallel programs we often write recursive functions, particularly
`divide-and-conquer' algorithms, to take into account how deep into a
computation a call is. This allows the program to avoid the creating of
parallel threads when the computation is obviously (to the programmer) not
worthwhile. An example of this was shown in Section \ref{sec:parAndSeq} with
our definition of \<quicksort\>. This pattern is not only limited to
parallelising the `shallower' levels of the computation and then switching to a
sequential version. It is possible that the leaves of a computation are the
expensive part of the computation, and to limit the number of generated
threads, you begin with a sequential version and switch to a parallelised
version below some depth. The general shape of the technique is drawn in Figure
\ref{fig:compDepth}.

\begin{figure}[t]
\centering
\begin{tikzpicture}[
    level 1/.style={sibling distance=6cm},
    level 2/.style={sibling distance=3cm},
    level 3/.style={sibling distance=1.5cm},
    level 4/.style={sibling distance=0.75cm}]
  \node [hassef] {}
    child{ node [hassef] {} 
            child{ node [hassef] {} 
                            child{ node [hasse] {}
                                            child{ node [hasse] {}}
                                            child{ node [hasse] {}}
                                 }
                            child{ node [hasse] {}
                                            child{ node [hasse] {}}
                                            child{ node [hasse] {}}
                                 }
            }
            child{ node [hassef] {}
                            child{ node [hasse] {}
                                            child{ node [hasse] {}}
                                            child{ node [hasse] {}}
                                 }
                            child{ node [hasse] {}
                                            child{ node [hasse] {}}
                                            child{ node [hasse] {}}
                                 }
            }               
    }
    child{ node [hassef] {}
            child{ node [hassef] {} 
                            child{ node [hasse] {}
                                            child{ node [hasse] {}}
                                            child{ node [hasse] {}}
                                 }
                            child{ node [hasse] {}
                                            child{ node [hasse] {}}
                                            child{ node [hasse] {}}
                                 }
            }
            child{ node [hassef] {}
                            child{ node [hasse] {}
                                            child{ node [hasse] {}}
                                            child{ node [hasse] {}}
                                 }
                            child{ node [hasse] {}
                                            child{ node [hasse] {}}
                                            child{ node [hasse] {}}
                                 }
            }
    }
  ;
  \draw [cyan, dashed, ultra thick] (-5.2,-3.6) -- (5.2,-3.6);
  \node [above, cyan] at (0,-3.6) {$d$};
\end{tikzpicture}
\caption{A tree representation of a recursive computation}
\label{fig:compDepth}
\end{figure}

Given some depth of the computation $d$ (usually measured in the number of
recursive calls) for the function \<f\>, the shaded nodes represent calls to one
specialised form of \<f\> and the unshaded nodes represent another specialisation
of \<f\>. We can attain these two version simply with the following transformation

\begin{haskell}
f x = \hsinf{\<e\>} \(\phantom{space}\) \Longrightarrow
\(\phantom{space}\) &f_{1} d x &= \hsif{d > depth}{%
                                                     f x}{%
                                                     \ \hsinf{\<e_{[f \(\mapsto\) f1 (d + 1)]}\>}} \\
\end{haskell}

All original calls to \<f\> in the source program become \<f_{1} 0\> and only
below a certain depth do we begin to call the original function. What benefit does
this give us? Now any \verb|par| sites in \<f\> are duplicated in \<f_{2}\>
and can be switched independently based on the depth. This gives the iterative
step the flexibility to switch off the \verb|par|s for the levels of the call
tree that are not worthwhile to perform in parallel.

In order to choose an appropriate value for \<depth\> in \<f_{1}\> the runtime
system must be equipped with some proxy for call-depth. This could be in a form
similar to what is used to get stack traces for lazy functional languages
\citep{AllwoodStack}.

We hypothesise that the value of \<depth\> in \<f_{1}\> does not have to be
perfect on the first attempt. By ensuring that a \emph{reasonable} \<depth\> is
chosen the compiler can then determine which \verb|par|s, the ones in \<f_{1}\>
or the ones lower in the call-depth in \<f\>. Once the on/off decisions have
been made the compiler could attempt to \emph{tune} the value of \<depth\>.

\subsection{Knowing When to Specialise on Depth}

A subtle point is knowing \emph{when} this form of specialisation is useful.
As with the main argument of this thesis, we would use both static and dynamic
information about the program.  We believe that candidate functions for
specialisation on depth must exhibit \emph{at least} the following properties:

\begin{enumerate}
    \item The function must be recursive
    \item The function must parallelise a recursive call to itself
    \item The \verb|par| site accomplishing the above has a wide distribution
            of \verb|par| health (as exemplified by \verb|par| site \#8 in
            \ref{fig:sumHist})
\end{enumerate}

We believe that for large scale programs, specialising on depth will be
necessary as divide-and-conquer algorithms are quite common.
