Before we discuss the techniques used for disabling the parallelism that is too
costly, we must first discuss \emph{how} we determine which \verb|par| sites
are not worthwhile.  This involves recording a significant amount of runtime
information and a method for safely switching off the \verb-par- annotations.

As mentioned earlier, our runtime system is designed in the tradition of
projects like GranSim \citep{gransim}. The goal is to have as much control of
the execution substrate as possible. This allows us to investigate certain
trade-offs while ensuring that we minimise confounding variables.

\subsection*{Logging}

The runtime system maintains records of the following global statistics:

    \begin{itemize}
        \item Number of reduction cycles
        \item Number of sparks
        \item Number of blocked threads
        \item Number of active threads
    \end{itemize}

These statistics are useful when measuring the overall performance of a
parallel program, but tell us very little about the usefulness of the threads
themselves.

In order to ensure that the iterative feedback system is able to determine the
overall `health' of a thread, it is important that we collect some statistics
pertaining to each individual thread. We record the following metrics for each
thread:

    \begin{itemize}
        \item Number of reduction cycles
        \item Number of sparks generated
        \item Number of threads blocked by this one
        \item Which threads have blocked the current thread
    \end{itemize}

This allows us to reason about the productivity of the threads themselves.  An
ideal thread will perform many reductions, block very few other threads, and be
blocked rarely. A `bad' thread will perform few reductions and be blocked for
long periods of time.

\subsection{Adjusting the Cost of Parallelism}

Because our simulator abstracts away from many of the bookkeeping details
of the runtime system the creation and management of a thread is very close
to free. In fact, the creation of a parallel task only costs the time
of the \verb|par| function itself! Only a handful of instructions!

This is clearly too optimistic. In order to better model the fact that
creating and managing parallel tasks incurs \emph{real} cost, we must
implement a method of simulating this overhead.

\todoinline{Writer more here and show plots of different simulated overheads}
