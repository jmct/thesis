We are now able to describe the last piece of the puzzle. Once we have
recorded our runtime feedback we must then \emph{modify} the amount of
parallelism in the program. There are at least two suitable methods for
accomplishing this task:

    \begin{enumerate}
        \item Have the compiler use the feedback before bytecode generation
            and transform the program accordingly
        \item Incorporate the feedback by modifying the bytecode itself
    \end{enumerate}

We have chosen the latter option as it suits our needs and provides sufficient
foundation for further exploration. That being said there is an explicit
trade-off that has occured\todo{wording}.

\subsection{Switchable \texttt{par}s}
\label{sec:switchPar}

    In order to take advantage of runtime profiles we must be able to adapt the
compilation based on any new information.  One choice is to recompile the
program completely and create an oracle that uses the profiles. This way the
oracle can better decide which subexpressions to parallelise. Our approach is to
modify the runtime system so that it is able to \emph{disable} individual
\verb-par- annotations. When a specific \verb=par= in the source program is
deactivated it no longer creates any parallel tasks while still maintaining the
semantics of the program.  The method has two basic steps:

    \begin{itemize}
        \item \verb=par='s are identified via the $G$-Code instruction
                \verb=PushGlobal "par"= and each par is given a unique identifier.
        \item When a thread creates the heap object representing the call to
                \verb=par= the runtime system looks up the status of the \verb=par= using its
                unique identifier. If the \verb=par= is `on' execution follows as normal. If the
                \verb=par= is `off' the thread will ignore the $G$-Code instruction \verb=Par=.
    \end{itemize}

