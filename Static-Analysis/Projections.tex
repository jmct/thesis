In this section we provide a brief overview of the two
predominant techniques for strictness analysis: \emph{abstract interpretation}
and \emph{projection-based} analysis.  We then motivate our decision to use a
projection-based analysis.

\subsection{Projections}

This explosion in cost motivated Wadler and Hughes to propose using
\emph{projections} from domain theory to analyse strictness
\citep{wadler1987projections}.


Projection-based analysis provides two benefits over abstract interpretation:
the ability to analyse functions over arbitrary structures, and a
correspondence with parallel strategies \citep{marlow2010seq, strategies}. This
allows us to use the projections provided by our analysis to produce an
appropriate function to compute the strict arguments in parallel.

Strictness analysis by abstract interpretation asks ``When passing $\bot$ as an
argument is the result of the function call $\bot$?''. Projection-based
strictness analysis instead asks ``If there is a certain degree of demand on
the result of this function, what degree of demand is there on its
arguments?''.

What is meant by `demand'? As an example, the function \verb'length' requires
that the input list be finite, but no more. We can therefore say that
\verb'length' \emph{demands} the spine of the argument list. The function
\verb'append' is a more interesting example:

\begin{alltt}
        append :: [a] -> [a] -> [a]
        append []     ys = ys
        append (x:xs) ys = x : append xs ys
\end{alltt}

By studying the function we can tell that the first argument must be defined to
the first cons, but we cannot know whether the second argument is ever needed. However, what
if the \emph{result} of \verb'append' needs to be a finite list? For example:

\begin{alltt}

    lengthOfBoth :: [a] -> [a] -> Int
    lengthOfBoth xs ys = length (append xs ys)
\end{alltt}

In this case \emph{both} arguments to \verb'append' must be finite. Projections
can be used to formalise this type of context \citep{wadler1987projections,
hinze1995projection}.

\subsection{Semantics of Projections}

Given a domain $D$, a projection on $D$ is a continuous function
$\pi \ : \ D \rightarrow D$ that satisfies

\begin{align}
\pi \sqsubseteq ID \\
\pi \circ \pi = \pi
\end{align}

Equation (3) ensures that a projection can not add any information to a value,
i.e. all projections approximate the identity function. Idempotence (4) ensures
that projecting the same demand twice on a value has no additional effect. This
aligns with our intuition of demand. If we demand that a list is spine-strict,
demanding spine-strictness again does not change the demand on the list.

Because we want the introduction of parallelism to be semantics-preserving we
use the following safety condition for projections:

\begin{equation}
\gamma \ \circ \ f = \gamma \ \circ \ f \ \circ \ \pi
\end{equation}

Given a function $f \ : X \rightarrow Y$, and demand $\gamma$ on the
\emph{result} of $f$, projection-based analysis propagates the demand given by
$\gamma$ to the arguments of $f$. This results in the demand on the
\emph{arguments} of $f$ given by $\pi$.  The analysis aims to find the
\emph{smallest} $\pi$ for each $\gamma$, but approximating towards $ID$ (as
it is always safe to project the identity).

\paragraph{Demands on Primitives}
On unlifted base types, such as unboxed integers, there are two demands,
$ID$ and $BOT$, with the following semantics


\begin{align}
ID \ x \ &= \ x \\
BOT \ x \ &= \ \bot
\end{align}


When an expression is in a $BOT$ context it means that non-termination is
inevitable. You can safely evaluate an expression in this context because there
is no danger of \emph{introducing} non-termination that is not already present.

\paragraph{Demands on Lifted Types} Haskell's non-strict semantics means that
most types we encounter are \emph{lifted} types.  Lifted types represent
possibly unevaluated values. Given a demand $\pi$ on $D$, we can form two
possible demands on $D_{\bot}$, $\pi!$ and $\pi?$; strict lift and lazy lift
respectively. To paraphrase Kubiak et al.: $\pi!$ means we will definitely need
the value demanded by this projection, and we will need $\pi$'s worth of it
\citep{kubiak}. $\pi?$ does not tell us whether we need the value or not, but if
we \emph{do} need the value, we will need it to satisfy $\pi$'s demand.

\paragraph{Demands on Products} A projection representing a demand on a product
can be formed by using the $\otimes$ operator with the following semantics

\begin{align*}
\langle \pi_{1} \otimes \dots \otimes \pi_{n} \rangle \ \bot &= \bot \\
\langle \pi_{1} \otimes \dots \otimes \pi_{n} \rangle \ 
\langle x_{1}, \dots, x_{n} \rangle &= \langle \pi_{1} x_{1}, \dots, \pi_{n} x_{n} \rangle
\end{align*}

\paragraph{Demands on Sums} If projections are functions on a domain, then
$\oplus$, the operator that forms projections on sum-types performs the case-analysis.

\begin{align*}
[ID_{True} \oplus ID_{False}]  \ True &= True \\
[ID_{True} \oplus BOT_{False}] \ False &= \bot
\end{align*}

\begin{figure}
\begin{align*}
    d ::=&\ BOT              & \text{Bottom (hyperstrict)} \\
        |&\ ID               & \text{Top (the identity)} \\
        |&\ \langle d_{1} \otimes d_{2} \dots \otimes d_{n} \rangle   & \text{Products} \\ 
        |&\ [d_{1} \oplus d_{2} \dots \oplus d_{n}]    & \text{Sums} \\ 
        |&\ \mu\beta . d     & \text{Recursive Demands} \\
        |&\ d?               & \text{Strict Lift} \\
        |&\ d!               & \text{Lazy Lift}
\end{align*}
\caption{Abstract Syntax for Contexts of Demand}
\label{fig:ContextAST}
\end{figure}


Figure \ref{fig:ContextAST} presents a suitable abstract syntax for projections
representing demand.  This form was introduced by Kubiak et al. and used in
Hinze's work on projection-based analyses \citep{kubiak, hinze1995projection}.
We have omitted the details on the representation of context variables (for
polymorphic demands), for a comprehensive discussion we suggest Chapter 6 of Hinze's
dissertation \citep{hinze1995projection}.

In short, projections representing demand give us information about how defined
a value must be to satisfy a function's demand on that value. Knowing that a
value is definitely needed, and to what degree, allows us to evaluate the value
before entering the function.

\subsection*{Example Projections}

Because our primitives can be modelled by a flat domain (just $ID$ and $BOT$),
our lattice of projections corresponds with the two-point domain used in
abstract interpretation.

\hfill$\Box$

For pairs of primitive values, possible contexts include:
\begin{align}
[\langle ID? \otimes ID? \rangle] \label{IDPairs} \\
[\langle ID! \otimes ID? \rangle] \label{FSTPairs}
\end{align}


As Haskell's types are sums of products, pairs are treated as sums with only
one constructor.  For product types each member of the product is lifted.
Context \ref{IDPairs} is the top of the lattice for pairs, accepting all
possible pairs. Context \ref{FSTPairs} requires that the first member be
defined but does not require the second element. This is the demand that
\verb-fst- places on its argument.

\hfill$\Box$

For polymorphic lists there are 7 principal contexts; 3 commonly occurring contexts are:

\begin{align}
    \mu\beta.&[ID \oplus \langle \gamma? \otimes \beta?\rangle] \label{IDList} \\
    \mu\beta.&[ID \oplus \langle \gamma? \otimes \beta!\rangle] \label{FINList} \\
    \mu\beta.&[ID \oplus \langle \gamma! \otimes \beta!\rangle] \label{FULLList}
\end{align}


Here $\mu$ binds the name for the `recursive call' of the projection and
$\gamma$ is used to represent an appropriate demand for the element type of the
list.  An important point is that this representation for recursive contexts
restricts the representable contexts to \emph{uniform projections}: projections
that define the same degree of evaluation on each of their recursive components
as they do on the structure as a whole. The detailed reason for this
restriction is given on page 89 of Hinze \citep{hinze1995projection}. This
limitation does not hinder the analysis significantly as many functions on
recursive structures are themselves uniform.

With this in mind Context \ref{IDList} represents a lazy demand on the list,
Context \ref{FINList} represents a \emph{tail strict} demand, and Context
\ref{FULLList} represents a \emph{head and tail} strict demand on the list.

\hfill$\Box$

It will be useful to have abbreviation for a few of the contexts on lists. These
abbreviation are presented in Figure \ref{contexts}.

\begin{figure}[h!]
\begin{itemize}
    \item[] ID: accepts all lists
    \item[] T (tail strict): accepts all finite lists
    \item[] H (head strict): accepts lists where the head is defined
    \item[] HT: accepts finite lists where every member is defined
\end{itemize}
\caption{Four contexts on lists as described in \citep{wadler1987projections}.}
\label{contexts}
\end{figure}

We can now say more about the strictness properties of \verb'append'. The
strictness properties of a function are presented as a \emph{context
transformer} \citep{hinze1995projection}. 

\begin{align*}
    &append(ID) &\rightarrow &&ID!&;ID? \\
    &append(T)  &\rightarrow &&T!&;T! \\
    &append(H)  &\rightarrow &&H!&;H? \\
    &append(HT) &\rightarrow &&HT!&;HT!
\end{align*}

This can be read as ``If the demand on the result of \verb-append- is $ID$
then the first argument is strict with the demand $ID$ and the second
argument is lazy, but if it \emph{is} needed, it is with demand $ID$.

\hfill$\Box$

Following Hinze \citep{hinze1995projection} we construct projections
for every user-defined type. Each projection represents a
specific strategy for evaluating the structure, as we shall define in section
\ref{sec:derivation}. This provides us with the ability to generate
appropriate parallel strategies for arbitrary types. Using a
projection-based strictness analysis, we avoid the exponential blowup
of domains required for abstract interpretation.
