As mentioned in the Introduction \todo{Check where that was} the majority of
functional languages use either call-by-need or call-by-value semantics.  While
call-by-need has many attractive properties the delaying of computation incurs
an overhead cost on all computations. Call-by-value, in contrast, has an
execution model that is more easily mapped to conventional hardware, allowing
for simpler implementations that achieved good performance \tocite{SPJ or
someone}. Mycroft used this tension to motivate his development of strictness
analysis \tocite{mycroft thesis}:

\begin{displayquote}

The above arguments suggest that call-by-value is more efficient but
call-by-need preferable on aesthetic/definedness considerations. So techniques
are herein developed which allow the system to present a call-by-need interface
to the user but which performs a pre-pass on his program annotating those
arguments which can validly be passed using call-by-value.

\end{displayquote}

By determining which arguments can be safely passed using call-by-value
we diminish the overhead of call-by-need, paying the overhead of
suspending computation only when necessary to ensure that call-by-need
semantics are maintained.

While this was the original motivation for strictness analysis it also serves
in identifying potential parallelism in a program. When an argument is suitable
to be passed as call-by-value it is also suitable to be evaluated in parallel.
For this use the suspension becomes a future \tocite{futures}, the value is
evaluated in parallel to the original thread, synchronisation is then
accomplished via the same mechanism as laziness except a thread can be blocked
while waiting for another thread to complete its evaluation.

\subsection{Safety First}

Strictness analysis is chiefly concerned with \emph{safety}. In order to retain
the origin call-by-need semantics the runtime can only alter the evaluation order
when doing so guarantees the same termination properties of the program.

We will refer to this notion of safety as the \emph{strictness} properties
of an argument. Take a function $f$ of $n$ arguments

\begin{equation*}
f \ x_{1} \ \dots x_{i} \ \dots \ x_{n} \ = \langle \texttt{function body} \rangle
\end{equation*}

The $i$th argument of $f$ is said to be strict \emph{if and only if}

\begin{equation}
f \ x_{1} \ \dots \bot_{i} \ \dots \ x_{n} \ =  \bot
\end{equation}
\label{eq:idealSafety}

For any possible values of $x_{1}-x_{i-1},x_{i+1}-x_{n}$\footnote{Throughout this
dissertation $\bot$ is used to represent all forms of non-termination.}.

Equation \ref{eq:idealSafety} can be read as ``$f$ is strict in $x_{i}$ when $f$
fails to terminate if $x_{i}$ fails to terminate''. The reason this allows us
to evaluate the $i$th argument before its needed is because doing so would only result
in introducing non-termination if the program would have resulted in non-termination
otherwise.

\subsection{Abstract Domains}

Now that we have established what it means to be strict we can expand on how we
analyse programs for this property. As with any abstract interpretation
\todo{ensure this term is defined} this involves the choice of an abstract domain.

%% Hasse diagram for the flat domain of integers.
%%
%% We have a horizontal 'number line' from -infity to infinity and edges from each
%% number to _|_ which is centered below the numbers.
\begin{figure}[!h]
\centering
\begin{tikzpicture}
    \node [hasse, label=above:\large$0$]                (zero) at (0,0) {};
    \node [hasse, right = of zero, label=above:\large$1$] (one)   {};
    \node [hasse, right = of one, label=above:\large$2$]  (two)   {};
    \node [hasse, left = of zero, label=above:\large$-1$]  (neg1)  {};
    \node [hasse, left = of neg1, label=above:\large$-2$]  (neg2)  {};
    \node [hasse, left = of neg2, label=above:\large$\dots$]  (ldots) {};
    \node [hasse, right = of two, label=above:\large$\dots$]  (rdots) {};
    \node [hasse, left = of ldots, label=above:\large$-\infty$] (nInf)  {};
    \node [hasse, right = of rdots, label=above:\large$\infty$](inf)   {};

    \node [hasse, below = of zero, label=below:\large$\bot$] (bot)   {};

    \draw[black] (zero) -- (bot);
    \draw[black] (one) -- (bot);
    \draw[black] (two) -- (bot);
    \draw[black] (neg1) -- (bot);
    \draw[black] (neg2) -- (bot);
    \draw[black] (ldots) -- (bot);
    \draw[black] (rdots) -- (bot);
    \draw[black] (inf) -- (bot);
    \draw[black] (nInf) -- (bot);
\end{tikzpicture}
\caption{Flat Domain}
\label{fig:flatInts}
\end{figure}

In non-strict languages types like integers and Booleans form a flat domain;
either we have a value of that type, or we have $\bot$. This is depicted in
Figure \ref{fig:flatInts}. We can form an intuition of these orderings by
thinking about how much we \emph{know} about a certain value. While the
interger value $5$ maybe greater than the integer value $4$, we know the
same amount about each of them: their values. However, if we have a procedure
that is meant to compute an integer, and therefore has the type \<Int\>, and it
loops forever, we cannot know know that integer's value. Therefore we know less
about a non-terminating value.


This fits nicely with call-by-need semantics: an argument to a function of
type \<Int\> is really a computation that can either result in a value, or
result in non-termination. In terms of strictness analysis, this allows us to
abstract our real domain of integers to the simple two-point domain shown in
figure \ref{fig:twoPointNice}.

This is the domain we use for basic strictness analysis. The bottom of the
lattice, $\bot$, as implied above, represents \emph{definitely non-terminating}
expressions. The top of the lattice, $\top$, is used to represent
\emph{potentially terminating}\footnote{Remember that program analysis must
approximate in the general case.} expressions. This approximation is can seem
counterintuitive; why are we allowing the analysis to say some results are
potentially terminating when they could be non-terminating? The reasoning is
that non-termination is okay under non-strict semantics! If we approximated in
the opposite direction (as analyses for other purposes sometimes do) we may
accidentally compute a value that was never needed, defeating the purpose of
call-by-need evaluation.


\begin{figure}
\centering
\begin{tikzpicture}
    \node [hasse, label=above:\large$\top$]                 (top) {};
    \node [hasse, below = of top, label=below:\large$\bot$] (bot) {};

    \draw[black] (top) -- (bot);
\end{tikzpicture}
\caption{Two-point Domain}
\label{fig:twoPointNice}
\end{figure}

\subsubsection{Abstracting Functions}

Now that we know what it means to be strict and why we represent flat domains
as a two-point domain the next step is to abstract the functions in our
program. 

The idea is simple: for every function in our program of type \<A\> we must
produce a function of type \<A^{\#}\> that works on the abstracted
values\footnote{Some texts represent an abstracted type by a number \<N\> where
$N$ is the number of points in the abstract domain. We prefer to retain the
context of where this abstract domain came from.}. This abstracted program
is then interpreted using an abstract semantics that provides us with the
strictness information for each function in our program.

\todoinline{standard abstract interpretation diagram}


While this separation of abstracting the program and then performing an
abstract interpretation is useful from the theoretical point of view, many
compilers skip the intermediate representation of an abstracted program and
perform the abstract interpretation with the original AST
\citep{hinze1995projection, kubiak, SergeyDemand}.

We begin with the set of primitive arithmetic functions. In the case of F-Lite,
each numeric primitive is strict in both arguments, providing us with the
following for each of \<(+), (-), (*), (/)\>:

\begin{haskell*}
(\(\odot\)) &::& Int^{\#} -> Int^{\#} -> Int^{\#} \\
\(\top\) \(\odot\) \(\top\) &=& \(\top\) \\
\(\top\) \(\odot\) \(\bot\) &=& \(\bot\) \\
\(\bot\) \(\odot\) \(\top\) &=& \(\bot\) \\
\(\bot\) \(\odot\) \(\bot\) &=& \(\bot\)
\end{haskell*}

We must also be able combine results from different paths in a program. This
requires both conjunction and disjunction. We can use the \<meet\> ($\sqcap$)
and \<join\> ($\sqcup$) from our lattice which are fully described in Figure
\ref{fig:twopointMeet}.

\begin{figure}
\centering
\begin{minipage}{.5\textwidth}
\begin{haskell*}
\(\top\) \(\sqcap\) \(\top\) &=& \(\top\) \\
\(\top\) \(\sqcap\) \(\bot\) &=& \(\bot\) \\
\(\bot\) \(\sqcap\) \(\top\) &=& \(\bot\) \\
\(\bot\) \(\sqcap\) \(\bot\) &=& \(\bot\)
\end{haskell*}
\end{minipage}
\quad\quad
\begin{minipage}{.5\textwidth}
\begin{haskell*}
\(\top\) \(\sqcup\) \(\top\) &=& \(\top\) \\
\(\top\) \(\sqcup\) \(\bot\) &=& \(\top\) \\
\(\bot\) \(\sqcup\) \(\top\) &=& \(\top\) \\
\(\bot\) \(\sqcup\) \(\bot\) &=& \(\bot\)
\end{haskell*}
\end{minipage}
\caption{The \emph{meet} ($\sqcap$) and \emph{join} ($\sqcup$) for our lattice}
\label{fig:twopointMeet}
\end{figure}

We can now define an abstract interpretation, $\mathcal{A}$, that takes
expressions in our language and gives us their abstracted values. We require an
environment that maps variables and functions to abstracted values, we use,
\<\hasphi :: Id \to Exp^{\#}\> to represent this environment. We write
\<\hasphi[x \mapsto v]\> to represent extending the environment with identifier
\<x\> being mapped to the value \<v\>. Lastly, looking up a value in the environment
is just applying the environment to the identifier.

\begin{figure}
\begin{haskell*}
\mathcal{A} &::& Exp \to Env^{\#} \to Exp^{\#} \\
%
\mathcal{A}\Sem{Var v} \hasphi &=& \hasphi v \\
%
\mathcal{A}\Sem{Int i} \hasphi &=& \(\top\) \\
%
\mathcal{A}\Sem{Con c} \hasphi &=& \(\top\) \\
%
\mathcal{A}\Sem{Fun f} \hasphi &=& \hasphi f\\
%
\mathcal{A}\Sem{App (Fun f) [a_{1}, \dots a_{n}]} \hasphi &=&
        \mathcal{A}\Sem{f} \hasphi (\mathcal{A}\Sem{a_{1}} \hasphi) \dots
        (\mathcal{A}\Sem{a_{n}} \hasphi)\\
%
\mathcal{A}\Sem{Let b_{1} = e_{1} \(\dots\) b_{n} = e_{n} in e} \hasphi &=&
        \mathcal{A}\Sem{e} \hasphi[b_{i} \(\mapsto\) \mathcal{A}\Sem{e_{i}}] \\
%
\mathcal{A}\Sem{Case e alts}    \hasphi &=& \mathcal{A}\Sem{e} \hasphi
        \(\sqcap\) \mathcal{C}\Sem{alts} \hasphi \\
%
\quad&\quad&\quad \\
%
\hsnoalign{\mathcal{C}\Sem{alt_{1}, alt_{2}, \(\dots\), alt_{n}} \hasphi =
        \mathcal{A}\Sem{alt_{1}} \hasphi \(\sqcup\)
        \mathcal{A}\Sem{alt_{2}} \hasphi \(\sqcup \dots\) \(\sqcup\)
        \mathcal{A}\Sem{alt_{n}} \hasphi}
\end{haskell*}
\caption{An Abstract Semantics for Strictness Analysis on a Two-Point Domain}
\label{fig:twoPointAI}
\end{figure}

\todoinline{Fix rule C so that we take into account the vars introduced by case
alternatives (they should all be approximated as Top)}

\subsection{Some Examples}

We can now use the abstract semantics from Figure \ref{fig:twoPointAI} on some
real functions.

\subsubsection{The Constant Function:}

The function \<const\> is defined as

\begin{haskell*}
const x y = x
\end{haskell*}

For non-recursive functions, like \<const\>, we can determine the strictness
properties fully with just $n$ iterations of $\mathcal{A}$ where $n$ is the
number of arguments to the function. We run the abstract interpretation using
$\bot$ as the value for the argument we are currently interested in and $\top$
for all the rest.

First we analyse \<x\>:
\begin{haskell*}
\mathcal{A}\Sem{x}[x \mapsto \(\bot\), y \mapsto \(\top\)] \Rightarrow \(\bot\)
\end{haskell*}

then \<y\>:
\begin{haskell*}
\mathcal{A}\Sem{x}[x \mapsto \(\top\), y \mapsto \(\bot\)] \Rightarrow \(\top\)
\end{haskell*}

Remembering what it means to be strict from Equation \ref{eq:idealSafety}, this
analysis tells us that \<const\> is strict in \<x\> but not in \<y\>. This is
exactly what we would expect.

\hfill$\Box$

\subsubsection{Conditional}

In non-strict languages we can define our own control-flow abstractions,
allowing what is usually a primitive, the \<if\> statement, to be defined
naturally in the language as

\begin{haskell*}
if p t e = \hscase{p}{\\
                      True \to t\\
                      False \to e}
\end{haskell*}

Analysing \<if\> should determine that is strict in \<p\>

\begin{haskell*}
\mathcal{A}\Sem{Case\ p\  [True \to t, False \to e]} \hasphi
\hsbody{%
    \quad& \Rightarrow \mathcal{A}\Sem{p} \hasphi \(\sqcap\)
        (\mathcal{A}\Sem{t} \hasphi \(\sqcup\)
         \mathcal{A}\Sem{e} \hasphi)\\
    \quad& \Rightarrow \(\bot \sqcap (\top \sqcup \top)\)\\
    \quad& \Rightarrow \(\bot\)
}
\hswhere{%
\quad \hasphi = [p \mapsto \(\bot\),t \mapsto \(\top\), e \mapsto \(\top\)]
}
\end{haskell*}

This shows how the use of \<meet\> and \<join\> are used to combine the
results from the different branches of the function. Because the discriminant
of a \<Case\> expression is always evaluated to WHNF, a non-terminating
discriminate results in a non-terminating function.

Notice that if we analysed the second two arguments to \<if\> then we would
have seen that there are not strict for \<if\> because only one would be $\bot$
at a given time and \<meet\> ($\sqcup$) only results in bottom if both arguments
are bottom.


\hfill$\Box$

\subsubsection{Calling Abstract Functions}

Calling functions in the abstract interpretation is the same as a function
call in the standard interpretation except that the target of the call is
the abstracted function. Dependency analysis is used to ensure that callees
are analysed before their callers. The following example illustrates calling
functions and the fact that the abstraction of \<Case\> is capable of determining
when an argument is needed in all the branches.


\begin{haskell*}
addOrConst b x y = \hscase{b}{\\
    True \to x + y\\
    False \to x
    }
\end{haskell*}

The analysis for the second argument proceeds as follows

\begin{haskell*}
\mathcal{A}\Sem{Case\ b\  [True \to x + y, False \to x]} \hasphi
\hsbody{%
    \quad& \Rightarrow \mathcal{A}\Sem{b} \hasphi \(\sqcap\)
        (\mathcal{A}\Sem{x + y} \hasphi \(\sqcup\)
         \mathcal{A}\Sem{x} \hasphi)\\
    \quad& \Rightarrow \(\top\) \(\sqcap\)
        (\mathcal{A}\Sem{x + y} \hasphi \(\sqcup\)
         \mathcal{A}\Sem{x} \hasphi)\\
    \quad& \Rightarrow \(\top\) \(\sqcap\)
        ((\mathcal{A}\Sem{+} \hasphi (\mathcal{A}\Sem{x} \hasphi) (\mathcal{A}\Sem{y} \hasphi)) \(\sqcup\)
         \(\top\))\\
    \quad& \Rightarrow \(\top\) \(\sqcap\)
        ((+^{\#} \(\bot\) \(\top\)) \(\sqcup\)
         \(\top\))\\
    \quad& \Rightarrow \(\top \sqcap (\bot \sqcup \top)\)\\
    \quad& \Rightarrow \(\bot\)
}
\hswhere{%
\quad \hasphi = [b \mapsto \(\top\),x \mapsto \(\bot\), y \mapsto \(\top\)]
}
\end{haskell*}

The other language constructs are interpreted similarly. Do note that we do not
attempt to analyse functions with free variables, instead we take advantage of
lambda-lifting in order to remove nested functions to the top level. We are not
the first to use lambda lifting in order to avoid this problem
\citep{clack1985strictness}. Luckily lambda lifting is done regardless for
compilation to the $G$-Machine.

\hfill$\Box$


\subsubsection{Recursive Functions}

Our last concern is with recursive functions. We use the fact that recursive
abstract functions are just recursive functions one a different domain. This
allows us to define a recursive function as the least upper bound of successive
approximations of the function, i.e. an \emph{ascending Kleene chain} (AKC).

We calculate this by starting with the `bottom' approximation where all sets of
inputs are mapped to $\bot$. For a function $f$ we call this $f^{\#0}$. We then
calculate $f^{\#n}$ by replacing each call to $f$ with $f^{\#(n - 1)}$. This
series of successive approximations forms the AKC.

So for a function of the form

\begin{haskell*}
f x_{1} \dots x_{m} = \hsinf{body of \<f\> which includes a call to \<f\>}
\end{haskell*}

We generate the following AKC

\begin{haskell*}
f^{\#0} x_{1} \dots x_{m} &=& \(\bot\) \\
f^{\#1} x_{1} \dots x_{m} &=& \hsinf{body of \<f\> with call to \<f^{\#0}\>} \\
f^{\#2} x_{1} \dots x_{m} &=& \hsinf{body of \<f\> with call to \<f^{\#1}\>} \\
f^{\#3} x_{1} \dots x_{m} &=& \hsinf{body of \<f\> with call to \<f^{\#2}\>} \\
\quad &\vdots& \quad \\
f^{\#n} x_{1} \dots x_{m} &=& \hsinf{body of \<f\> with call to \<f^{\#(n - 1)}\>}
\end{haskell*}

We can stop the calculation of this AKC when $f^{\#n} \equiv f^{\#(n - 1)}$ for
\emph{all combinations} of $x_{1}$ to $x_{m}$. This means the for each iteration
of the AKC we must interpreter $f$ $2^m$ times!

Luckily there are clever ways of avoiding much of this expense for the average
cases. Clack and Peyton Jones developed an algorithm for the efficient
calculation of an AKC \citep{clack1985strictness}.

\hfill$\Box$

\subsubsection{Discussion of Two-Point Strictness Analysis}

We have now seen how to use rule $\mathcal{A}$ from Figure \ref{fig:twoPointAI}
to analyse functions in our programs. Now we can see how the results help us in
our ultimate aim of implicit parallelism. For this discussion we will forget
that parallelism is not always beneficial and focus on how we would utilise
all the possible parallelism.

Imagine we have a function $f$ with a call to $g$ in its body.

\begin{haskell*}
f \dots = \dots g e_{1} e_{2} e_{3} \dots
\end{haskell*}

Out analysis may determine that $g$ is strict in its first two arguments,
providing us with an opportunity for parallelism. This would allow us
to safely rewrite $f$ as the following

\begin{haskell*}
f \dots = \dots \hslet{x &= e_{1}\\
                       y &= e_{2}
                       }{%
                       x `par` y `par` g x y e_{3} \dots
                       }
\end{haskell*}

The expressions $e_{1}$ and $e_{2}$ are bound to the names $x$ and $y$
in a \<let\> expression so that the results of parallel evaluation are
shared with the body of $g$.

As mentioned above, the two-point domain informs us about strictness up to
WHNF.  sSo if $e_{1}$ or $e_{2}$ are values from a flat domain, this
transformation will provide all the benefit possible to $g$\footnote{Again,
ignoring the fact that it may not actually be a positive benefit.}. But what if
these arguments are of a non-flat type, like pairs or lists?

Because we aim for safe parallelism, we cannot evaluate $e_{1}$ or $e_{2}$ any
further than WHNF, already eliminating a vast quantity of potential parallelism!
Take the function \<sum\> for example:

\begin{haskell*}
f \dots = \dots \hslet{xs &= e_{1}
                       }{%
                       xs `par` sum xs \dots
                       }
\end{haskell*}

When a programmer wants to express this idiom in their code they often use
\emph{parallel strategies} as illustrated in Section \ref{sec:Approaches}.
This allows the programmer to write an expression similar to \<xs `using`
parList\>. This forces the evaluation of the list beyond WHNF. Because our
two-point domain does not guarantee the safety of evaluating beyond WHNF we are
not able to use a strategy like \<parList\>. This means that even though we
`know' that \<sum\> requires the list fully, the analysis has no way to
represent this, and can only determine that the outermost constructor is
needed!

Because of this shortcoming, strictness analysis in this form is inappropriate
for discovering the parallelism in a program that uses nested structures. In
the next section we will see how this deficiency is overcome by choosing
suitable domains for non-flat structures.

%
%data Exp &=& App Exp [Exp] \\ &|& Case Exp [Alt] \\ &|& Let [Binding] Exp \\
%&|& Fun Id \\
