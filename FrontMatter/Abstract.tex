We propose a new technique for exploiting the inherent parallelism in lazy
functional programs. Known as \emph{implicit parallelism}, the goal of writing
a sequential program and having the compiler improve its performance by
determining what can be executed in parallel has been studied for many years.
Our technique abandons the idea that a compiler should accomplish this feat in
`one shot' with static analysis and instead allow the compiler to
\emph{improve} upon the static analysis iterative feedback.

We demonstrate that iterative feedback can be relatively simple when the source
language is a lazy purely functional programming language. We present two main
methods of feedback to the compiler: \emph{Black box} feedback and
\emph{profile directed} feedback. These allow the compiler to base its
adjustments on overall runtime alone and with access to runtime profile data
respectively.
