As the speed of processors is starting to plateau, chip manufacturers are
instead looking to multi-core architectures for increased performance. The
ubiquity of multi-core hardware has made parallelism an important tool in
writing performant programs. Unfortunately, parallel programming is still
considered an advanced technique and most programs are written as sequential
programs.

We propose that we lift this burden from the programmer and allow the compiler
to automatically determine which parts of a program can be executed in
parallel. Historically, most attempts at auto-parallelism depended on static
analysis alone. While static analysis is often able to find \emph{safe}
parallelism, it is difficult to determine \emph{worthwhile} parallelism. This
is known as the \emph{granularity problem}. Our work shows that we can use
static analysis \emph{in conjunction with} search techniques by having the
compiler execute the program and then alter the amount of parallelism based on
execution speed.  We do this by annotating the program with parallel annotations
and using search to determine which annotations to enable.

This allows the static analysis to find the safe parallelism and shift the
burden of finding worthwhile parallelism to search. Our results show that by
searching over the possible parallel settings we can achieve better performance
than static analysis alone.
