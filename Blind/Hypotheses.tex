Because the two algorithms we are exploring have stochastic aspects it is
important that we are careful in measuring the results and ensuring that we
proper statistical methods when deciding whether our technique `works'.

Part of this process is deciding on what we are testing \emph{before} we run
the experiments and perform any statistical analysis. This ensures that we
compare the appropriate statistics.

\paragraph{}

The main thesis of our work is that by adding an iterative step to the
automatic parallelisation we get better performance than through static
analysis alone. Therefore the most important question to test is the following:

\paragraph{RQ1} What speed-up is achieved by using search to enable a
subset of \verb-par-s compared to the enabling all the \verb-par-s found by
static analysis?\\


\noindent
While the previous question is important, it is also important that we gain
speedups \emph{as compared to the sequential version}. For that reason it is
important to ask the following question:

\paragraph{RQ2} What speed-up is achieved by parallelisation using search
compared to the sequential version of the software-under-test (SUT)?\\


\noindent
As discussed in the last section, we consider two algorithms: a simple
hill-climbing algorithm and a greedy algorithm:

\paragraph{RQ3} Which search algorithm achieves the larger speed-ups, and
how quickly do these algorithms achieve these speed-ups?\\

\noindent
Because we have decided to use two methods for the initial \verb|par| settings
we have one final research question:

\paragraph{RQ4} Which form of initialisation enables the algorithm to
find the best speed-ups: all \verb-par-s enabled (we refer to this as
\emph{`all-on'} initialisation), or a random subset enabled (\emph{`random'}
initialisation)?
