Because we have chosen to perform the iterative step of compilation after
bytecode generation, we can know \emph{statically} how many \verb|par|s are in
a program. This allows us to represent a specific \emph{setting} of a program's
parallelism by a bitstring of static length, allowing us to perform standard
search techniques. The motivation for doing so is simple, while our platform
can record runtime statistics and profiling data, the overall goal is to
produce a program that performs better than the input program. The collection
of profiling data is simply one means of accomplishing this (which we will
investigate in Chapter \ref{chap:prof-search}). Using the target goal, the
improved performance of our program, as a fitness function allows this method
to be used even when a runtime system is not able to record such profiling
data.

There are a number of heuristic search techniques that perform search space
exploration using the evaluation of a fitness function
\citep{russell1995artificial}. For this thesis we have chosen to explore two
algorithms: A greedy algorithm, guaranteed to take linear time in the number of
bits, and a hill-climbing algorithm that explores more of the search space but
at the cost of being \emph{potentially} exponential in the number of bits.

\paragraph{Representation} 

We represent the choice of enabled \verb-par-s as a bit string where a 1
indicates that the \verb-par- is applied at a site, and 0 that it is not. The
length of the bit string is the number of \verb-par-s annotations introduced by
the static analysis and transformation stage of the compiler.

Each index points to a unique address in the bytecode of the program, and the
order of bits does not have any semantic meaning, i.e. the bit at index $1$
does does not necessarily have any relationship to the bit at index $2$.

\paragraph{Fitness}

For both algorithms we use the same fitness function: the overall runtime of
the program (measured in bytecode reductions). Therefore we aim to
\emph{minimize} the result of a fitness evaluation.

\subsection{Greedy Algorithm}

The greedy algorithm is designed to be simple but effective. The intuition is
that each bit is either beneficial to the program's performance, or
detrimental. Therefore the greedy algorithm visits each bit \emph{exactly
once}, by never revisiting a bit after a decision has been made about it we can
guarantee a linear time complexity.

The greedy algorithm considers the bits in our representation in a random
order.  This avoids any potential bias towards the early bits in a bitstring.
As each bit is considered, the bit is flipped from its current setting and the
program is evaluated using the settings of the resulting bitstring; the setting
of the bit---current or flipped---with the better fitness is retained.  The
algorithm terminates once all the bits have been evaluated.


\begin{algorithm}
\DontPrintSemicolon
\TitleOfAlgo{Greedy bitstring search}
\SetKwData{Setting}{setting}
\SetKwData{Fitness}{fitness}
\SetKwData{Best}{best}
\SetKwArray{Set}{setting}
\SetKwFunction{uniqueRand}{uniqueRand}
\SetKwFunction{evaluateProg}{evaluateProg}
\SetKwFunction{flipSwitch}{flipSwitch}

\KwData{An initial \texttt{par} setting as a bitstring of size $N$}
\KwResult{The best performing bitstring}
\BlankLine

\Best.\Fitness $\leftarrow \infty$
\BlankLine

\For{$i \leftarrow 0$ \KwTo $N$}{
    $j \leftarrow$ \uniqueRand{$N$}\;
    \flipSwitch{\Set{$j$}}\;
    \BlankLine
    \Fitness $\leftarrow$ \evaluateProg{}\;
    \BlankLine
    \eIf{\Fitness $<$ \Best.\Fitness}{
        \Best.\Fitness $\leftarrow$ \Fitness\;
        \Best.\Setting $\leftarrow$ \Setting\;
    }{
        \flipSwitch{\Set{$j$}}\;
    }
}
\BlankLine
\Return \Best.\Setting\;
\caption{Greedy \texttt{par}-Setting Search}
\label{list:greedy}
\end{algorithm}

The listing in Algorithm \ref{list:greedy} provides the actual algorithm used
in performing our greedy search. There are a few points to be aware of:

\begin{itemize}
  \item The function \FuncSty{uniqueRand} provides random numbers but ensures
        that the same index will not be visited twice.
  \item The function \FuncSty{evaluateProg} runs the loaded bytecode file,
        returning the global reduction count
  \item \FuncSty{flipSwitch} modifies the bytecode pointed at by an index so that
        the function calls switches from \verb|par| to \verb|parOff| or vice versa
  \item \DataSty{best} is a structure containing a fitness value and a \verb|par|
        setting
\end{itemize}

\subsection{Hill-Climbing Algorithm}

We utilise a simple hill-climbing algorithm in which the neighbours of the
current bitstring are those formed by flipping a single bit.  At each
iteration, these neighbours of the current bitstring are considered in a random
order, and the fitness evaluated for each in turn. The first neighbour that has
a better fitness, i.e.~fewer reductions are made by the main thread, than the
current bitstring becomes the current bitstring in the next iteration.  The
algorithm terminates when no neighbour of the current bitstring has a better
fitness.

\begin{algorithm}
\DontPrintSemicolon
\TitleOfAlgo{Hill-Climbing bitstring search}
\SetKwData{Setting}{setting}
\SetKwData{Fitness}{fitness}
\SetKwData{Best}{best}
\SetKwData{True}{True}
\SetKwData{False}{False}
\SetKwData{Searching}{searching}
\SetKwArray{Set}{setting}
\SetKwFunction{uniqueRand}{uniqueRand}
\SetKwFunction{evaluateProg}{evaluateProg}
\SetKwFunction{Refresh}{refreshUniques}
\SetKwFunction{flipSwitch}{flipSwitch}

\KwData{An initial \texttt{par} setting as a bitstring of size $N$}
\KwResult{The best performing bitstring}
\BlankLine

\Best.\Fitness $\leftarrow \infty$
\BlankLine

\While{searching}{
    \Searching $\leftarrow$ \False\;
    \BlankLine
    \BlankLine

    \For{$i \leftarrow 0$ \KwTo $N$}{
        $j \leftarrow$ \uniqueRand{$N$}\;
        \flipSwitch{\Set{$j$}}\;
        \BlankLine
        \Fitness $\leftarrow$ \evaluateProg{}\;
        \BlankLine
        \eIf{\Fitness $<$ \Best.\Fitness}{
            \Best.\Fitness $\leftarrow$ \Fitness\;
            \Best.\Setting $\leftarrow$ \Setting\;
            \BlankLine
            \Searching $\leftarrow$ \True\;
            \BlankLine
            break\;
        }{
            \flipSwitch{\Set{$j$}}\;
        }
    }
    \Refresh{}\;
}
\BlankLine
\Return \Best.\Setting\;
\caption{Hill-Climbing \texttt{par}-Setting Search}
\label{list:hill-climb}
\end{algorithm}

Our hill-climbing algorithm (shown in Algorithm \ref{list:hill-climb}) is only
slightly more complex than our greedy algorithm, but much more powerful. Its
power comes from its ability to revisit indices in the bitstring. The function
\FuncSty{refreshUniques} the state within \FuncSty{uniqueRand} to allow for
already visited indices to be generated again. However it does \emph{not} reset
the key in the psuedorandom number generator.

\subsection{Initial \texttt{par} Setting}

While we now have both algorithms in hand, there is still an important decision
to be made. As shown in the algorithm listings, neither algorithm initialises
the bitstring representing our \verb|par| setting. There are three obvious
choices available to us:

\begin{enumerate}
    \item All bits on
    \item All bits off
    \item A random bitstring
\end{enumerate}

Based on our problem domain, option $1$ seems the most intuitive. We would like
to begin with as much parallelism as possible, and prune out detrimental
\verb|par|s. For this reason, we will not consider option $2$. However, it is
possible that by starting in a random location in the search space, we may
avoid a local optimum that our search techniques may encounter. So we
experiment with both an initial setting with all \verb|par|s on, and with
random initial settings\footnote{We realise that by being random we actually
include option $2$ as well, but it becomes a rare edge-case}.
