Having explored the design of our experimental platform we can now begin
describing some of the experiments that we have conducted. In this chapter we
use standard bitstring search techniques during our iterative step, i.e. we do
not utilise \emph{any} runtime information other than the actual \emph{running
time} of the program.

The intuition is that because each \verb|par| site is a switch, our technique
lends itself to representing the `setting' of our parallel program as a
bitstring, each index in the string representing a single \verb|par|. The
fitness function, in our case, is the program's execution time. The healthier a
\verb|par| setting, the faster our program will run.

\subsection*{Plan of the Chapter}

The rest of this chapter describes our technique in more detail. Section
\ref{sec:blind-ParFunc} presents the two heuristic algorithms that we will use
and the expected trade-offs for each. We describe our empirical method and
results in Section \ref{sec:blind-Results}. Lastly, we offer our conclusions
and discuss related work in Section \ref{sec:blind-Conclusion}.
