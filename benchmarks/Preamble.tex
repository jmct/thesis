In this section we will give a brief introduction to the benchmark programs
that are used in experimenting with our platform. We have also provided the source
listing for each program in Appendix \ref{append:bench}.

\paragraph{SumEuler}
SumEuler is a common parallel functional programming benchmark first introduced
with the work on the $\langle\nu, G\rangle$-Machine in 1989 \citep{vGMachine}.
This program is often used as a parallel compiler benchmark, making it a
`sanity-check' for our work. We expect to see consistent speed-ups in this
program when parallelised (9 \verb-par- sites).

\paragraph{Queens and Queens2}
We benchmark two versions of the nQueens program. Queens2 is a purely symbolic
version that represents the board as a list of lists and does not perform
numeric computation (10 \verb-par- sites for Queens and 24 for Queens2). The
fact that Queens2 has more than double the number of \verb-par- sites for the
same problem shows that writing in a more symbolic style provides more
opportunity for \emph{safe} parallelism.

\paragraph{SodaCount} The SodaCount program
solves a word search problem for a given grid of letters and a list of
keywords.  Introduced by Runciman and Wakeling, this program was chosen because
it exhibits a standard search problem and because Runciman and Wakeling
hand-tuned and profiled a parallel version, demonstrating that impressive
speed-ups are possible with this program \citep{Runciman:1996:AFP:242105} (15
\verb-par- sites).

\paragraph{Tak}
Small recursive numeric computation that calculates a Takeuchi number. Knuth
describes the properties of Tak in \citep{ExamplesOfRecursion} (2 \verb-par-
sites).

\paragraph{Taut}
Determines whether a given predicate expression is a tautology. This program
was chosen because the algorithm used is \emph{inherently sequential}. We feel
that it was important to demonstrate that not all programs have implicit
parallelism within them, sometimes the only way to achieve parallel speed-ups
is to rework the algorithm (15 \verb-par- sites).

\paragraph{MatMul}
List of list matrix multiplication. Matrix multiplication is an inherently
parallel operation; we expect this program to demonstrate speed-ups when
parallelised (7 \verb-par- sites).
