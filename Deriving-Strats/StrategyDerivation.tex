\section{Deriving Strategies from Projections}
\label{sec:proAndStrat}

One of the reasons that projections were chosen for our strictness analysis is
their correspondence to parallel strategies.  Strategies are functions whose
sole purpose is to force the evaluation of specific parts of their arguments
\citep{marlow2010seq, strategies}. All strategies return the unit value
\verb-()-. Strategies are not used for their computed result but for the
evaluation they force along the way.

\subsection*{Some Examples}

The type for strategies is defined as \verb`type Strategy a = a -> ()`.

The simplest strategy, named \verb-r0- in \citep{marlow2010seq}, which performs
no reductions is defined as \verb-r0 x = ()-. The strategy for weak head normal
form is only slightly more involved: \verb-rwhnf x = x `seq` ()-

\hfill$\Box$

The real power comes when strategies are used on data-structures. Take
lists for example. Evaluating a list sequentially or in parallel provides us
with the following two strategies

\begin{alltt}
    seqList s []     = ()
    seqList s (x:xs) = s x `seq` (seqList s xs)

    parList s []     = ()
    parList s (x:xs) = s x `par` (parList s xs)
\end{alltt}

Each strategy takes another strategy as an argument. The provided
strategy is what determines how much of each element to evaluate. If the provided
strategy is \verb-r0- the end result would be that only the spine of the list is
evaluated. On the other end of the spectrum, providing a strategy that evaluates a value
of the list-item type \verb-a- fully would result in list's spine \emph{and} elements being evaluated.

\hfill$\Box$

Already we can see a correspondence between these strategies and the contexts shown
in Figure \ref{contexts}. The $T$ context (tail strict) corresponds to the strategy
that only evaluates the spine of the list, while the $HT$ context corresponds to the
strategy that evaluates the spine and all the elements of a list.

Recalling the program in Figure \ref{sumLast} the generated function
\verb-mainListS1- corresponds to a strategy for evaluating a list in a
$HT$ context. In our derived strategies it is not necessary to pass
additional strategies as arguments because the demand on a structure's elements
makes up part of the context that describes the demand on that structure.

\subsection*{Derivation Rules}

%    & \mathcal{C} :: \Sem{Context} \rightarrow Names \rightarrow Exp \\
%    & \mathcal{A}\Sem{(Constr, Context)} \rightarrow Names \rightarrow (Pat, Exp) \\

\begin{figure}[t]
\begin{displaymath}
  \begin{aligned}
   \noalign{$\mathcal{C} :: \Sem{Context} \rightarrow Names \rightarrow Exp $}
    & \mathcal{C}\Sem{c?}    \phi  &&= \lambda x \rightarrow () \\
    & \mathcal{C}\Sem{c!}     \phi  &&= \Sem{c} \phi \\
    & \mathcal{C}\Sem{\mu\beta.c} \phi  &&= fix \ (\lambda n \rightarrow \Sem{c} (n:\phi)) \\
    & \mathcal{C}\Sem{\beta}       (n:\phi) &&= n \\
    & \mathcal{C}\Sem{[cs]}     \phi  &&= \lambda x \rightarrow Case \ x \ of \ \mathcal{A}\Sem{cs} \phi \\
    & \mathcal{C}\Sem{c}                      \phi  &&= \lambda x \rightarrow x \  `seq` \ () \\
    & \quad && \\
    \noalign{$\mathcal{A} :: \Sem{(Constructor, Context)} \rightarrow Names \rightarrow (Pat, Exp) $}
    & \mathcal{A}\Sem{(C_{n}, ID)} \phi &&= (C_{n}, ()) \\
    & \mathcal{A}\Sem{(C_{n}, BOT)}        \phi &&= (C_{n}, ()) \\
    & \mathcal{A}\Sem{(C_{n}, \langle cs \rangle)} \phi &&= (C_{n} \ vs, \mathcal{F}\Sem{ss} \phi) \\
    \noalign{$\qquad where \enspace ss = filter \ (isStrict \circ fst) \ \$ \ zip \ cs \ vs $} 
    \noalign{$\qquad \hphantom{where} \enspace vs = take \ (length \ cs) freshVars $}
    & \quad && \\
    \noalign{$\mathcal{F} :: \Sem{[(Context, Exp)]} \rightarrow Names \rightarrow Exp $}
    & \mathcal{F}\Sem{[]} \phi           &&= () \\
    & \mathcal{F}\Sem{((c, v):[])} \phi  &&= App \ (Fun \ ``seq") \ [App \ (\mathcal{C}\Sem{c} \phi)\  [v], ()] \\
    & \mathcal{F}\Sem{((c, v):cs)} \phi  &&= App \ (Fun \ ``par") \ [App \ (\mathcal{C}\Sem{c} \phi)\  [v], ls] \\
    \noalign{$\qquad where \enspace ls =  \mathcal{F}\Sem{cs} \phi$}
  \end{aligned}
\end{displaymath}
\caption{Rules to generate strategies from demand contexts}
\label{demStrat}
\end{figure}

Because projections \emph{already} represent functions on our value domain, translating
a projection into a usable strategy only requires that we express the projection's
denotation in a programmed definition. The rules we use are shown in Figure \ref{demStrat}.
Rule $\mathcal{C}$ constructs a strategy for all of the context constructors except for
products. This is because product types are only found within constructors in the
source language and are therefore wrapped in sums as constructor-tag context
pairs. These pairs are handled by the $\mathcal{A}$ rule.



One aspect of strategies that does \emph{not} directly correspond to a context
is the choice between \verb-seq- and \verb-par-. Every context can be fully
described by both sequential and parallel strategies. When a constructor has
two or more fields, it can be beneficial to evaluate \emph{some} of the fields
in parallel. It is not clear, generally, which fields should be evaluated in
parallel and which should be evaluated in sequence. As shown in rule $\mathcal{F}$
we evaluate all fields in parallel \emph{except} for the last field in a structure.
This means that if a structure has only one field then its field will be evaluated
using \verb-seq-.

\subsection{Specialising on Demand}

The key reason for performing a strictness analysis in our work is to know when
it is \emph{safe} to perform work before it is needed. This work can then be
sparked off and performed in parallel to the main thread of execution. Using
the projection-based analysis allows us to know not only \emph{which} arguments
are needed, but \emph{how much} (structurally) of each argument is needed. We
convert the projections into strategies and then spark off those strategies in
parallel.

Assume that our analysis determines that function \verb`f` is strict in both of
its arguments.  This allows us to convert

\begin{verbatim}
        f e1 e2
\end{verbatim}

into

\begin{verbatim}
        let a = e1
            b = e2
        in (s1 a) `par` (s2 b) `seq` (f a b)
\end{verbatim}

where \verb`s1` and \verb`s2` are the strategies generated from the projections on those
expressions.


\subsection*{Different Demands on the Calling Function}

If a function has different demands on its result at different calling sites,
that is dealt with `for free' using the transformation above. However, there
may be multiple \emph{possible} demands \emph{at the same call site}.

This can happen when there are different demands on the calling function, for
example:

\begin{verbatim}
        func x = f e1 e2
\end{verbatim}

Different demands on the result of \verb`func` may mean different demands on
the result of \verb`f`. This in turn means that different transformations would
be appropriate. Assume this results in having two different demands on
\verb`f`. One demand results in the first transformation (\verb-funcD1-) and
the other results in the second (\verb-funcD2-). How do we reconcile this
possible difference?

\subsection*{Specialisation by Demand}

To accommodate this possibility we can clone the function \verb`func`. One
clone for each demand allows us to have the `more parallel' version when it
is safe, and keep the `less parallel' version in the appropriate cases. Note,
we do not have to clone all functions with versions for every possible demand.
Instead we can do the following for each function:

\begin{enumerate}
    \item Determine which demands are actually present in the program
    \item In the body of the function, do the different demands result in differing
        demands for a specific function call?
    \item If no, no cloning
    \item If yes, clone the function for each demand and re-write the call-sites to call
        the appropriate clone
\end{enumerate}

Applying the above procedure to our hypothetical expression would result in the
following


\begin{verbatim}
funcD1 x = let a = e1
               b = e2
            in s1 a `par` s2 b `seq` f a b

funcD2 x = let a = e1
           in s1 a `par` f a e2
\end{verbatim}


