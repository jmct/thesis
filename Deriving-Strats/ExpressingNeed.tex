Burns introduced the idea of \emph{evaluation transformers} as a way to specify
how much of a value can be reduced safely before it is needed
\citep{burn1987evaluation}. By using the results of a four-point strictness
analysis Burns was able to annotate expressions based on how much of the
structure could be safely evaluated. The main insight was that each annotation
represented a demand on the values and that demand could be propagated to the
sub-expressions in a well defined way.

Burns defined four `evaluators', each for lists, with the following meanings


\begin{itemize}
    \item $\xi_{0}$ performs no evaluation
    \item $\xi_{1}$ evaluates its argument to WHNF
    \item $\xi_{2}$ evaluates the spine of its argument
    \item $\xi_{3}$ evaluates the spine of its argument and each element to WHNF
\end{itemize}

These correspond directly to each point in Wadler's four-point domain. The
compiler can apply the relevant evaluation transformer to each expression that
it is safe to do so. Burns realised that the strictness of certain functions
can be dependent on which evaluation transfomer forced the evaluation of the
function itself. With hindsight we can see that this is very similar to the
motivation behind projection based analysis (and indeed Burns' noted this
relationship in later work \tocite{Burns' paper on projections and
ETs}.Additionally, the runtime system can keep track of which evaluation
transformer is used on each expression. This allows the runtime system to
propogate the evaluation where possible.

There are two downsides with this approach: the level of evaluation is limited
to pre-determined types (in this case lists and basic values) and it is the
\emph{runtime system} that determines how the evaluation transformers are
propogated. Ideally the propagation of demand would be static so that the
runtime system would not have the additional bookeeping and management involved
in the method introduced by Burns.
