The information that projection-based strictness analysis provides us is
concerned with how defined a value must be for a function to be defined given a
certain demand on the result of that function. This is reflected in the
safety condition for this analysis, which we remind ourselves of below:

\begin{equation}
\gamma \ \circ \ f = \gamma \ \circ \ f \ \circ \ \pi
\end{equation}

The projection-based analysis attempts to determine the smallest $\pi$ that
retains the semantics of $f$ for a given $\gamma$. This tells use which
arguments, if any, are safe to evaluate before entering $f$. Our goal now is to
take a given $\pi$ and transform the call to $f$ so that as much evaluation as
$\pi$ allows is done in parallel.

This chapter presents our method of achieving this goal automatically. We
describe this process as the \emph{derivation} of parallel strategies from
projections representing demand on a value, and it forms a core part of our
contribution.

\subsection*{Plan of the Chapter}

We discuss some parallels with Burns' \emph{Evaluation Transformers} in Section
\ref{sec:expressingNeed} which can be seen as a limited, manual version of our
technique. We then present our derivation rules in Section
\ref{sec:derivation}.  Section \ref{sec:parPlacement} demonstrates how we
introduce the derived strategies to the input program.
