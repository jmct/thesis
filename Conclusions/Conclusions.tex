In Chapter \ref{chap:prof-search} we showed that it is possible to use
profiling data from the execution of functional programs to \emph{refine} the
parallelism that is introduced by a static analysis. Importantly, despite being
a bitstring search, we were able to ensure that on average the search time is
sub-linear to the amount of parallelism (in the form of \verb|par| annotations)
that is introduced.

However, not all programs perform well and for this technique to work more
generally it is clear that some form of \emph{speculative} parallelism is
needed. This is made clear by the disappointing performance of the
\verb-MatMul- program.

Additionally, when all things are considered our method restricts the use of
the profiling information passed back to the compiler after an iteration. The
compiler can use profiling data to make decision about \emph{already existing}
\verb|par| sites, but can not decide to \emph{create} new \verb|par| sites or
transform the program further. There is no fundamental reason this can not be
done, but it would require a re-working of the static analysis and
transformation phases of the compiler (Chapter \ref{chap:discovery} and
\ref{chap:derivation}) to somehow incorporate that information. Doing so, while
being more flexible, is quite more complex than our approach of simply
modifying the bytecode produced by the compiler.
\\[0.35cm]
Chapter \ref{chap:blind} searched over bitstrings using only the total running
time of the program's execution. We experimented with two heuristic search
algorithms, a greedy search that only considered each bit in the bitstring once
and a traditional hill-climbing algorithm. This technique showed that even
without the ability to measure detailed profiling information, iterative
compilation can be beneficial. The downside of this simplicity is the
additional number of iterations that must be performed. Using the simpler
greedy algorithm allows us to cap the number of iterations to the number of
\verb|par| sites introduced, but as we saw with the \verb|SodaCount| program,
this can result in finding non-optimal (but pretty good) solutions.
\\[0.35cm]
Our experimental platform, as described in Chapter \ref{chap:platform}, is a
fairly standard implementation of a lazy functional language. The limiting
aspect of this implementation seems to be our need to defunctionalise. Our
higher-order specialisation is suitable for exploratory purposes but is not
appropriate for wide use. The Ur/Web compiler also requires a similar form of
defunctionalisation (i.e. not Reynold's style defunctionalisation)
\citep{ur-web} and has found success in using a modified version of a
call-pattern specialisation \citep{call-pattern-spec} to produce first-order
programs from higher-order programs. While admittedly not total, the author of
Ur/Web has stated that they are not aware of a single `real' program that this
technique is unable to accommodate \citep{chlipala}.
\\[0.35cm]
Chapter \ref{chap:derivation} introduces a method of automatically deriving
parallel strategies from arbitrary demands. A proof that our technique is total
and sound (all valid projections result in valid strategies) is desirable.
Before the ability to derive strategies parallelising programs automatically
was limited by the evaluation methods that were hard-coded into the compiler, often
based on Wadler's four-point domain \citep{hogen1992automatic, burn1987evaluation}.
This limited the applicability of these techniques to list-based programs, which
is an unnecessary limitation given the additional information that demand contexts
provide.

Despite the lack of a formal proof, we feel that this is a useful contribution
even outside the scope of implicit parallelism, we will discuss other
applications in Chapter \ref{chap:future}.

If we desire the ability to introduce new \verb|par|s (or other similar
extensions) during the iterative step (as mentioned above), this part of the
compiler would need to be adapted. One idea would be to change the order of the
\verb|par|s in a strategy to reduce thread collisions. Mostly, however, the
principal of translating a projection into a strategy would remain the same.
\\[0.35cm]
Implicit parallelism is often cited as a pipe-dream that is unlikely to provide
any real benefit to programmers \citep{spjQuote1999, marlowBook, benEmail,
hammond2000research}.  While we do not claim to have solved the problem
completely, we feel that we have demonstrated that the topic is worth pursuing,
and that in some cases a combination of static analysis and feedback-directed
compilation can achieve speedups `for free'. More importantly, we feel that
this thesis demonstrates that the topic is worth pursuing with fresh eyes.
Processor utilisation is much less important than it was when multi-processor
hardward was niche and expensive, pragmatic approaches that provide useful
speedups, even if not all for all programs, can be worthwhile in the multicore
era.
