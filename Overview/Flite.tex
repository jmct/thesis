The use of a small functional language as the internal representation of a
compiler is a common technique in functional compilers \cite{dutchBook,
PeytonJones:IFL, Augustsson:LazyMLCompiler, UHC}. By using a small core
language as an internal representation source language features are simply
syntactic sugar that is translated to a simpler but no less expressive
language. This provides compiler writers with a smaller surface area for
analysis and transformation. This has been used to great effect in the Glasgow
Haskell Compiler (GHC) which uses a small core language similar to ours
\cite{peyton2002secrets, jones1998transformation}. 

Functional languages vary widely in their syntax, features, and type systems,
but almost all functional languages are either strict (eager) or non-strict
(and usually lazy) in their evaluation model. It is important to understand the
distinction between these two systems. Because functional languages can be
seen as enriched lambda calculi, we can study different evaluation
orders\footnote{Many texts describe them as \emph{evaluation strategies}.  We
use the term order to avoid confusion with parallel strategies, which are a
different concept that play a central role in this thesis.} by demonstrating
them on a simple lambda calculus. There are four main evaluation orders that
can be used with the lambda calculus:

    \begin{enumerate}
        \item Call-by-value
        \item Call-by-name
        \item Call-by-need
        \item Normal-order
    \end{enumerate}

The differences between the first 3 can be easily illustrated using the
following function definitions:

\begin{align*}
    sqr \ x \  &= \  x * x \\
    bot \ \_ \ &= \  \bot
\end{align*}

Now assume we want to evaluate the expressions \<sqr (5*5)\> and \<bot (5*5)\>.
We can manually reduce each of these expressions using each of the evaluation
orders.

\paragraph{Call-by-value}

\begin{figure}[!h]
\centering
\begin{multicols}{2}
\noindent
\begin{align*}
     &sqr\ (5*5) \\
  =\ &sqr\ 25 \\
  =\ &let\ x\ =\ 25\ in\ x * x \\
  =\ &25 * 25 \\
  =\ &625
\end{align*}
\begin{align*}
     &bot\ (5*5) \\
  =\ &bot\ 25 \\
  =\ &let\ x\ =\ 25\ in\ \bot \\
  =\ &\bot
\end{align*}
\end{multicols}
\caption{Call-by-value reduction}
\label{fig:call-by-value}
\end{figure}

Note that the argument to \<sqr\> and \<bot\> is evaluated \emph{before}
we enter the function's body. 

\pagebreak

\begin{figure}[!h]
\centering
\begin{multicols}{2}
\noindent
\begin{align*}
     &sqr\ (5*5) \\
  =\ &let\ x \  =\ (5*5)\ in\ x * x \\
  =\ &let\ x \  =\ (5*5)\ in\ (5*5) * x \\
  =\ &let\ x \  =\ (5*5)\ in\ 25 * x \\
  =\ &25 * (5*5) \\
  =\ &25 * 25 \\
  =\ &625
\end{align*}
\begin{align*}
     &bot\ (5*5) \\
  =\ &let\ x\ =\ 5*5\ in\ \bot \\
  =\ &\bot
\end{align*}
\end{multicols}
\caption{Call-by-name reduction}
\label{fig:call-by-name}
\end{figure}

\paragraph{Call-by-name} Here reduction delays the evaluation of a function's
argument until its use.  However, the result of evaluating a value is not
shared with other references to that value. This results in computing \<5*5\>
twice.

\begin{figure}[!h]
\centering
\begin{multicols}{2}
\noindent
\begin{align*}
     &sqr\ (5*5) \\
  =\ &let\ x\ =\ 5 * 5\ in\ x * x \\
  =\ &let\ x\ =\ 25\ in\ x * x \\
  =\ &25 * 25 \\
  =\ &625
\end{align*}
\begin{align*}
     &bot\ (5*5) \\
  =\ &let\ x\ =\ 5*5\ in\ \bot \\
  =\ &\bot
\end{align*}
\end{multicols}
\caption{Call-by-need reduction}
\label{fig:call-by-need}
\end{figure}

\paragraph{Call-by-need} This is designed to avoid the duplication of work that
is often a result of call-by-name evaluation. Notice that in this evaluation
\<(5*5)\> is bound to \<x\> as before but the result of computing the value of
\<x\> the first time \emph{updates} the binding. This is why call-by-need is
often referred to as call-by-name \emph{with sharing}, or \emph{lazy}.

An important point is that for languages without arbitrary side-effects call-by-name
and call-by-need are semantically equivalent. Call-by-need is an optimisation in the
\emph{implementation} of reduction.


\begin{figure}[!h]
\centering
\begin{multicols}{2}
\noindent
\begin{align*}
     &sqr\ (5*5) \\
  =\ &let\ x\ =\ 5 * 5\ in\ x * x \\
  =\ &let\ x\ =\ 5 * 5\ in\ 25 * x \\
  =\ &let\ x\ =\ 5 * 5\ in\ 25 * 25 \\
  =\ &625
\end{align*}
\begin{align*}
     &bot\ (5*5) \\
  =\ &\bot
\end{align*}
\end{multicols}
\caption{Normal order reduction}
\label{fig:normal-order}
\end{figure}

\paragraph{Normal order} This method of evaluation is the only method that
obeys the semantic property that \<\(\lambda\) \_ \to \(\bot\) \(\equiv \
\bot\)\>.  This is because normal order reduction will evaluate under a lambda
\tocite{Abramsky's lazy lambda calculus paper}.

Of the four, only the first three are commonly used as the basis for
programming languages. Most languages are call-by-value, this includes
functional languages such as Scheme, OCaml, SML, and Idris. Fewer languages
are call-by-name, Algol 60 being the most notable case. Scala, while being
call-by-value by default, does allow programmers to specify that some
functions use call-by-name. Lastly, call-by-need is used by Haskell, Clean,
Miranda, and our own F-lite.

\todoinline{Explain Church-Rosser, at least a little bit}

The Church-Rosser Theorem gives us a profound guarantee with our
functional programs: Given a valid expression, there is only one normal form
for the expression. This is true regardless of the order of reductions carried
out (given that they are valid reductions). So given a program, there can be
many possible reduction orders that all lead to the same result. What does this
mean for sequential computation? Call-by-need evaluation For lazy languages,
such as Haskell, this means that there is no fear of only evaluating
expressions as they are needed and terminating with an incorrect result. There
is the following caveat: while there is \emph{only one} normal form, it is
possible that there could be a reduction order that does not terminate.

With the Church-Rosser theorem in hand does the evaluation order we choose
affect our aims with regard to automatic parallelisation? Systems designed to
take advantage of implicit parallelism have been written for languages that use
each of the three main evaluation orders \tocite{We can cite Jens Nicolay here
and some loop unrolling method too}. We have decided on call-by-need semantics
because it emphasises purity and has the sharing of computation built into the
execution model. The focus on purity allows the compiler to take certain
liberties with program transformation that may not otherwise be valid
\citep{jones1998transformation}. In the case of auto-parallelisation, we
are able to know that we could only alter the semantics of a program
by introducing non-termination. As we will see in Chapter \ref{chap:discovery}
there are methods to ensure we avoid this.

\paragraph{An aside} Many languages, including functional languages, that use
call-by-value semantics also provide the ability to perform arbitrary side
effects and mutation. This greatly hampers the feasibility of implicit
parallelisation because the \emph{sequence} of side-effects can alter the
semantics of the program. While programmers \emph{could} write pure programs
that do not rely on shared state, it is not enforced by the compiler as it is
for languages like Haskell. That being said there are techniques that can be
used to find safe parallelism in strict languages. \tocite{Matt Might's paper
at least}
