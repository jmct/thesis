This chapter is meant to accomplish two important goals: to provide a common
vocabulary for the rest of the thesis and to familiarise the reader with the
`gist' of our proposed technique. In regards to the first goal we will
introduce the syntax of F-lite and define terms that will be used throughout
the sequel. By offering an overview of our technique, the
reader will also possess context for each of the later chapters.

That being said, this chapter can be skipped if the reader is comfortable with
functional languages and compilers.

\subsection*{Plan of the Chapter}

The chapter begins by defining the syntax and semantics of F-lite and Folle in
Section \ref{sec:Flite} which are higher-order and first-order languages
respectively. Section \ref{sec:overview} presents a high-level view of our
compiler and its organisation.
